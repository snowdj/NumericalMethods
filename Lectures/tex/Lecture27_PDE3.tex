\documentclass{beamer}


\mode<presentation>
{
  \usetheme{Hawke}
  \setbeamercovered{transparent}
}

\usepackage[english]{babel}
\usepackage[latin1]{inputenc}
\usepackage{times}
\usepackage[T1]{fontenc}
\usepackage{multimedia}


%%%%%%
% My Commands
%%%%%%

\newcommand{\bb}{{\boldsymbol{b}}}
\newcommand{\bx}{{\boldsymbol{x}}}
\newcommand{\by}{{\boldsymbol{y}}}
\newcommand{\bfm}[1]{{\boldsymbol{#1}}}
\newcommand{\pda}[2]{\frac{\partial{#1}}{\partial{#2}}}
\newcommand{\pdb}[2]{\frac{\partial^2{#1}}{\partial{#2}^2}}
\newcommand{\pdc}[3]{\frac{\partial^2{#1}}{\partial{#2}\partial{#3}}}


%%%%

\title[Lecture 27] % (optional, use only with long paper titles)
{Lecture 27 - Von Neumann stability}

\author[I. Hawke] % (optional, use only with lots of authors)
{I.~Hawke}
\institute[University of Southampton] % (optional, but mostly needed)
{
  School of Mathematics, \\
  University of Southampton, UK
}

\date[Semester 1] % (optional, should be abbreviation of conference name)
{MATH3018/6141, Semester 1}

\subject{Numerical methods}

\pgfdeclareimage[height=0.5cm]{university-logo}{mathematics_7469}
\logo{\pgfuseimage{university-logo}}

\AtBeginSection[]
{
  \begin{frame}<beamer>
    \frametitle{Outline}
    \tableofcontents[currentsection]
  \end{frame}
}


\begin{document}

\begin{frame}
  \titlepage
\end{frame}

\section{Stability}

\subsection{Evidence}

\begin{frame}
  \frametitle{Partial differential equations}

  Partial differential equations (PDEs) involve derivatives of
  functions of more than one variable, say $u(x, y)$ or $y(t,
  x)$. Hence more complex behaviour and more interesting
  physics. \pause

  \vspace{1ex}

  Only look at linear problems.  Also only consider finite difference
  methods: simple to analyse but not always competitive. \pause

  \vspace{1ex}

  Previously looked at simple finite difference algorithms applied to
  heat and advection equations. At present have no explanation for
  which worked and which did not; for that we need stability theory.

\end{frame}

\begin{frame}
  \frametitle{The results}

  \begin{center}
    \begin{tabular}{c|c c}
      Algorithm & Heat equation & Advection equation \\ \hline
      FTCS & Stable $s < 1/2$ & Unstable \\
      BTCS & Stable $\forall s$ & ?? \\
      FTBS & ?? & Stable $c < 1$
    \end{tabular}
  \end{center}

  Here $s = k \delta / h^2$ for the heat equation
  \begin{equation*}
    \partial_t y - k \partial_{x x} y = 0,
  \end{equation*}
  and $c = v \delta / h$ for the advection equation
  \begin{equation*}
    \partial_t y + v \partial_x y = 0.
  \end{equation*}

\end{frame}


\subsection{Lax Theorem}

\begin{frame}
  \frametitle{Consistency}

  \begin{overlayarea}{\textwidth}{0.8\textheight}
    \only<1|handout:1>
    {
      Theory for multistep methods for IVPs related
      \emph{consistency}, \emph{stability} and \emph{convergence}.
      Have similar concepts and theory for PDEs.
    }
    \only<2-|handout:1->
    {
      Finite difference equation \emph{consistent} if it tends to the
      PDE as grid sizes go to zero \emph{independently}.
    }
    \only<3-|handout:1->
    {

      \vspace{1ex}
      Local error must vanish as $h, \delta \rightarrow 0$ in any way.
    }
    \only<4-|handout:1>
    {

      \vspace{1ex}
      All the methods defined previously are clearly consistent.
    }
    \only<5|handout:2>
    {

      \vspace{1ex}
      The Lax method for the advection equation,
      \begin{equation*}
        y_i^{n+1} = \frac{1}{2} \left( y_{i+1}^n + y_{i-1}^n \right) -
        \frac{c}{2} \left( y_{i+1}^n - y_{i-1}^n \right)
      \end{equation*}
      can be Taylor expanded to give
      \begin{equation*}
        \partial_t y + v \partial_x y = \tfrac{1}{2} \left(
          h^2 / \delta - v^2 \delta \right) \partial_{x x} y + \dots
      \end{equation*}
      which is consistent if, for example, $h / \delta$ is constant, but
      not in general.
    }
  \end{overlayarea}

\end{frame}

\begin{frame}
  \frametitle{Stability}

  As before a finite difference equation is \emph{stable} if it
  produces a bounded solution and unstable otherwise. \pause

  \vspace{1ex}

  Have seen that BTCS is stable for the heat equation, whilst FTCS was
  stable only when $s < 1/2$. For heat equation FTCS is
  \emph{conditionally} stable, whilst BTCS is \emph{unconditionally}
  stable. \pause Have also seen that FTCS is unstable for advection
  equation; it is \emph{unconditionally unstable}. \pause

  \vspace{1ex}

  As for IVPs there may exist genuine unbounded solutions, and
  spurious solutions that swamp genuine solutions.  Often closely
  related to the complex concept of well-posedness: shall ignore this
  issue here.

\end{frame}

\begin{frame}
  \frametitle{Convergence}

  Key concept is \emph{convergence}: in the limit of zero grid spacing
  the finite difference solution converges to the exact
  solution. \pause

  \vspace{1ex}

  Exceptionally difficult to directly prove in general. \pause Instead
  rely on

  \vspace{2ex}

  {\bf Lax's Theorem (1954)}: Given a properly posed linear
  initial-value problem and a finite difference approximation to it
  that is consistent, stability is necessary and sufficient for
  convergence. \pause

  \vspace{2ex}

  Can be extended (with some caveats) to nonlinear problems with
  non-trivial boundary conditions, even when discontinuities form.

\end{frame}


\section{Von Neumann stability theory}

\subsection{Von Neumann stability theory}

\begin{frame}
  \frametitle{Determining stability}

  Von Neumann stability method for linear algorithms and equations:
  \begin{enumerate}
  \item<2-> Make ansatz that solutions of the finite difference
    equation have form
    \begin{equation*}
      y_{\ell}^k =  e^{\alpha j \ell h} q^k, \qquad (\, j = \sqrt{-1}, \alpha \in {\mathbb R}, q \in {\mathbb C} ).
    \end{equation*}
  \item<3-> Substitute this solution into the finite difference
    equation.
  \item<4-> Determine $q$ as a function of $\alpha$, the grid spacings
    $h,\delta$, and known numbers (such as the advection velocity
    $v$).
  \item<5-> Determine under what conditions we have stability, i.e.\ when
    \begin{equation*}
      |q| < 1 \quad \forall \alpha.
    \end{equation*}
  \end{enumerate} \pause[6]
  Can apply method blindly to produce stability result; but how does
  it come about?

\end{frame}


\begin{frame}
  \frametitle{Von Neumann stability}

  \begin{overlayarea}{\textwidth}{0.8\textheight}
    \only<1-2|handout:1>
    {
      Consider a linear initial-value problem, such as heat or
      advection equation.  Assume we have a consistent finite
      difference algorithm (easily checked).
    }
    \only<2|handout:1>
    {

      \vspace{1ex}

      To prove convergence need to show algorithm is
      stable: i.e., numerical result is bounded.
    }
    \only<3-|handout:1->
    {
      Assume we can separate variables and write
      \begin{equation*}
        y = \sum_{m = 0}^{\infty} A_{m}(t) \cos (m x) +
        B_{m}(t) \sin (m x).
      \end{equation*}
    }
    \only<4-|handout:2>
    {
      Implies assumptions about e.g.\ spatial domain (ignore here).
    }
    \only<5-|handout:2>
    {

      \vspace{1ex}

      Use exponential form of trigonometric terms to
      write as
      \begin{equation*}
        y = \sum_{m = -\infty}^{\infty} a_{m}(t) e^{j m x}
      \end{equation*}
      where $a$ is \emph{complex} function (containing the
      two degrees of freedom in $A,B$), and $j = \sqrt{-1}$.
    }
    \only<6-|handout:2>
    {

      \vspace{1ex}

      As problem is linear can superpose solutions, so each
      $a_{m}$ is independent. Hence problem stable \emph{only
        if} $a_{m}$ does not grow with $t$ for \emph{every}
      $m$.  }
  \end{overlayarea}

\end{frame}

\begin{frame}
  \frametitle{Linear stability}

  % \begin{overlayarea}{\textwidth}{0.8\textheight}
  %   \only<1-4>
  %   {
      From linearity consider just the single mode ``solution''
      \begin{equation*}
        y = a_{m}(t) e^{j m x}, \qquad m \in \mathbb{Z}.
      \end{equation*}
      Just need to show that $y$ bounded for general $m$. \pause
    % }
    % \only<2-4>
    % {

      \vspace{1ex}

      Consider a point on grid $\{ x_{\ell} \} = \ell h$, so that
      \begin{equation*}
        y_{\ell} = a_{m}(t) e^{j m \ell h}.
      \end{equation*} \pause
    % }
    % \only<3-4>
    % {
      At first two steps $t^0, t^1 = t^0 + \delta$, have values
      \begin{equation*}
        y_{\ell}^0 = a_{m}(t^0) e^{j m \ell h}, \quad y_{\ell}^1 =
        a_{m}(t^1) e^{j m \ell h}.
      \end{equation*} \pause
    % }
    % \only<4>
    % {
      \emph{Define} the growth from one timestep to the next as
      \begin{equation*}
        q \equiv \frac{y_{\ell}^{k+1}}{y_{\ell}^k} =
        \frac{a_{m}(t^{k+1})}{a_{m}(t^k)} \quad \therefore \quad
        \text{Stability} \Leftrightarrow | q | < 1.
      \end{equation*}
  %   }
  % \end{overlayarea}

\end{frame}


\section{Von Neumann stability in practise}


\subsection{Von Neumann stability in practise}


\begin{frame}
  \frametitle{FTCS for the heat equation}

  Our finite difference algorithm is
  \begin{equation*}
    y_i^{n+1} = ( 1 - 2 s ) y_i^n + s \left( y_{i+1}^n + y_{i-1}^n
    \right).
  \end{equation*} \pause
  By identifying $i \leftrightarrow \ell$ and $n \leftrightarrow k$ we
  can substitute in our ansatz
  \begin{align*}
    && y_{\ell}^k & =  e^{\alpha j \ell h} q^k. \\
    & \implies &
    e^{\alpha j \ell h} q^{k+1} &= (1 - 2 s) e^{\alpha j \ell h} q^k +
    s \left( e^{\alpha j (\ell + 1) h} q^k + e^{\alpha j (\ell - 1) h}
      q^k \right).
  \end{align*} \pause
  Removing the common factor of $e^{\alpha j \ell h} q^k$ gives
  \begin{equation*}
    q = (1 - 2 s) + s \left( e^{\alpha j h} +  e^{-\alpha j h} \right).
  \end{equation*}
  Relates $q$ to (undetermined) $\alpha$ and (free) $h$ and $s$.

\end{frame}

\begin{frame}
  \frametitle{FTCS for the heat equation: 2}

  \begin{equation*}
    y_i^{n+1} = ( 1 - 2 s ) y_i^n + s \left( y_{i+1}^n + y_{i-1}^n
    \right).
  \end{equation*}
  \begin{equation*}
    q = (1 - 2 s) + s \left( e^{\alpha j h} +  e^{-\alpha j h} \right).
  \end{equation*}
  \begin{overlayarea}{\textwidth}{0.7\textheight}
    \only<1-2|handout:1>
    {
      Now bound $|q|$ for \emph{any} $\alpha$
      (may depend on grid spacing).
    }
    \only<2|handout:1>
    {
      More obvious by writing
      \begin{equation*}
        q = 1 - 4 s \sin^2 \left(\frac{\alpha h}{2} \right).
      \end{equation*}
    }
    \only<3-|handout:2>
    {
      \vspace{-1ex}
      \begin{equation*}
        q = 1 - 4 s \sin^2 \left(\frac{\alpha h}{2} \right)
      \end{equation*}
      See that, for arbitrary $\alpha$, $\sin$ term
      bounded between $0$ and $1$.
    }
    \only<4-|handout:2>
    {
      Therefore the limiting values for $q$ are
      \begin{align*}
        q & \le 1 & (\sin^2(\dots) & = 0) \\
        q & \ge 1 - 4 s & (\sin^2(\dots) & = 1).
      \end{align*}
    }
    \only<5-|handout:2>
    {
      Therefore stability, and hence convergence, is given when \vspace{-.5ex}
      \begin{equation*}
        |q| < 1 \Leftrightarrow s < \tfrac{1}{2} \Leftrightarrow
        \frac{\delta}{h^2} < \tfrac{1}{2}.
      \end{equation*}
    }
  \end{overlayarea}

\end{frame}

\section{Summary}

\subsection{Summary}

\begin{frame}
  \frametitle{Summary}

  \begin{itemize}
  \item Von Neumann stability analysis is the key tool for
    understanding the results seen ``experimentally'' in the last
    lecture.
  \item By looking at the growth or decay of individual frequency
    modes we can check the stability of the method; Lax's theorem then
    proves convergence.
  \item For complex equations and / or algorithms, it is often very
    difficult or impossible to apply even Von Neumann stability
    analysis.
  \item If an algorithm does not work in the linear case then it is
    highly unlikely to work for more complex problems!
  \end{itemize}

\end{frame}

\end{document}
